% !TeX root = ../main.tex

\ustcsetup{
  keywords = {123氧化锌,电子传输层,电激发瞬态吸收光谱,量子点发光二极管},
  keywords* = {zinc oxide, electron transport layer, Electrical excitation transient absorption spectroscopy, quantum dot light-emitting diode},
}

\begin{abstract}
  量子点发光二极管(QD-LEDs)因其高色彩饱和度、窄发射线宽、低成本溶液可加工性,以及在低驱动电压下实现高亮度和高外量子效率(EQE),在显示和照明应用中颇具吸引力。然而,QD-LEDs目前仍面临诸多挑战,包括在高电压电流下的效率滚降问题,蓝色 QD-LEDs 的短寿命和低外量子效率等。在QD-LEDs器件中,研究载流子(电子和空穴)的行为是非常关键的一环,已知的负面影响包括俄歇复合、焦耳热和漏电流等,而电子的传输途径涉及带间泄漏、直接重组(包括辐射和非辐射)、溢流泄漏和界面重组等。目前,对主流的氧化锌类电子传输层(ETL)机理的认识仍相对匮乏。因此,对氧化锌电子传输层的深入研究显得尤为重要。
  
  本文介绍了一种基于传统的泵浦-探测技术的电激发光探测的瞬态吸收光谱技术(EETA),用于研究电致发光器件工作过程中各功能层载流子的平衡浓度、载流子注入动力学及电场分布信息。我们搭建了紫外波段拓展的电激发瞬态吸收光谱仪,基于此研究目前主流的氧化锌(包括氧镁锌等)电子传输层QD-LED器件,对氧化锌电子传输层在QD-LED中机理进行深入研究。重点关注瞬态吸收光谱中350nm左右及之前的信号归属,以确定氧化锌电子传输层的信号以及其在不同样品、不同电压下的行为,这将为剖析氧化锌在QD-LED中的重要作用打下基础。

\end{abstract}

\begin{abstract*}
  Quantum dot light-emitting diodes (QD-LEDs) have garnered significant attention for display and solid-state lighting applications owing to the superior color purity, narrow emission linewidths, cost-effective solution processing, and demonstrated capacity for achieving high luminance with exceptional external quantum efficiency under low driving voltages. Nevertheless, critical performance limitations persist, particularly efficiency roll-off under high current densities, operational instability of blue light-emitting diodes, and suboptimal EQE in full-color displays. The key to solving these problems lies in understanding the carrier dynamics, especially the behaviors of electrons and holes. Negative effects include Auger recombination, Joule heating, and leakage currents, among which the electron transport involves complex processes, including interband leakage, direct recombination (both radiative and non-radiative), overflow leakage, and interfacial recombination. However, at present, the understanding of the ZnO-based electron transport layers (ETLs) is relatively scarce.
  
  This study introduces the electrically excited transient absorption spectrometer (EETA), a improved version of the pump-probe technique, to investigate the carrier dynamics of QD-LEDs. The EETA is enable to help analyze the equilibrium carrier concentration, carrier injection kinetics, and the electric field distribution between device layers. Through wavelength-resolved analysis of transient absorption signatures in the ultraviolet regime specific to ZnO-based electron transport layers,
  we can pinpoint the unique contribution of ZnO ETLs and shed light on the electron transport mechanisms. Through detailed examination of these signals across various samples and voltage conditions, we seek to clarify how ZnO ETLs affect electron injection, transportation, and recombination pathways. Ultimately, this work aims to help address efficiency roll-off, enhance device lifetimes, and improve EQE in QD-LEDs.
\end{abstract*}
